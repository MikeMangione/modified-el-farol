\documentclass[10pt]{article}
\usepackage[T1]{fontenc}
\usepackage{amssymb}
\usepackage{amsmath}
\usepackage{graphicx}
\usepackage{algpseudocode}
\usepackage{algorithm}
\usepackage{tikz}
\usetikzlibrary{arrows}
\usepackage{subfigure}
\usepackage{stackrel}
\usepackage{blindtext}
\oddsidemargin=0.15in
\evensidemargin=0.15in
\topmargin=-.5in
\textheight=9in
\textwidth=6.25in
\usepackage[colorlinks=true,breaklinks,pdfpagemode=none,linkcolor=blue,citecolor=blue]{hyperref}
\usepackage{enumerate}
\usepackage{enumitem}
\usepackage{amsmath,amsfonts,amssymb,bm}
\newenvironment{proof}{\noindent{\bf Proof}\hspace*{1em}}{\qed\bigskip}

\begin{document}
\section{Computational Game Theory Idea $\#1$ - Modified El Farol Bar Simulation:}

\subsection{Motivation:} 
While the standard El Farol premise presents many interesting problems,
\subsection{Simulation Elements}
The simulation will be comprised of three elements:
\begin{itemize}
\item (Potential) Patrons
\item Queue
\item Bar
\item Street (not in queue)
\end{itemize}

The Street, Que, and Bar act as containers for Patrons, and the Bar has a fixed occupancy limit. No Patron may exist in more than one environment at a given time. The Street and Queue are mutually visible, but the Bar is not visible to Patrons in the two other environments. \\
Patrons have a series of legal choices, dependent on the environment they exist in:
\begin{itemize}
\item Street:
\begin{itemize}
\item Enqueue
\item Pass Queue
\end{itemize}
\item Que:
\begin{itemize}
\item Wait In Queue
\item Leave Queue
\end{itemize}
\item Bar:
\begin{itemize}
\item Remain in Bar
\item Exit Bar
\end{itemize}
\end{itemize}
\subsection{High-Level Game Theory:}
Patrons have two metrics they will attempt to balance:\
\begin{itemize}
\item Happiness: Only increased when inside the bar
\item Time: Patrons will not wait in any environment indefinitely
\end{itemize}
Patrons will have individually allocated initial metric levels determined by two characteristics (existing on a continuum):
\begin{itemize}
\item Introversion v. Extroverision: Determines how much enjoyment is accrued in the bar, and what capacity will cause them to exit
\item Patience: Determines what queue the Patron will enter and how long they will remain before leaving
\end{itemize}
\subsection{Formalization}
For this simulation, we will assume that all Patrons have some desire to enter the bar (to some degree). A 0 person queue will cause all Patrons to enqueue. Patience for individual Patrons will exist in the space $P \in (0,100]$.
Introversion/Extroversion will affect the saturation tolerance of the Bar.\\
For Patron i, $T_i$ is the saturation tolerance such that M is maximum bar occupancy,\\ $X, Y \in [0,M], X \leq Y, T_i = [X/M,Y/M] $\\
We will likely want to create bounds on X and Y that actually respect the preferences of the Patron. It is unlikely an real person would be happy with both an empty bar and one filled to capacity.\\
I/E measures will represent a distribution centered around the ideal saturation for the individual. Distance from this ideal saturation will cause in a decrease in happiness returns for the individual, eventually resulting in decreasing happiness. Note that this has some interesting implications:
\begin{itemize}
\item High E Patrons will prefer saturation levels as close to occupancy as possible. Over time, this will result in a high occupancy bar being filled with High E Patrons as Lower E Patrons will leave.
\item Conversely, an unsaturated bar will actually cause High E unhappiness, and will cause these High E Patrons to leave. This will place Low E/ High I individuals at a majority, though it benefits a lower number of Is than the above E majority situation.
\end{itemize}





\end{document}




